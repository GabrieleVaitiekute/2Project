\chapter{2Project}
\hypertarget{md__r_e_a_d_m_e}{}\label{md__r_e_a_d_m_e}\index{2Project@{2Project}}
\label{md__r_e_a_d_m_e_autotoc_md0}%
\Hypertarget{md__r_e_a_d_m_e_autotoc_md0}%
Programa (vector) su meniu skirtingos programos eigos pasirinkimui\+: 1 -\/ ranka suvesti viską, 2 -\/ generuoti tik pažymius, 3 -\/ generuoti ir pažymius ir studentų vardus, pavardes, 4 -\/ nuskaityti duomenis iš failo (galima pasirinkti iš 4 skirtingų failų, programa apskaičiuoja ir pateikia, kiek laiko užtruko nuskaityti failą), 5 -\/ sugeneruoti failus (5 dydžių pasirinkimai), 6 -\/ baigti darbą, 7 -\/ testavimas.

Studentai suskirstomi į galvočius ir nepažangius pagal medianą arba vidurkį (leidžiama pasirinkti).

Programos eigoje suteikiama galimybė pasirinkti rikiavimą pagal vardą, pavardę, galutini balą, apskaičiuotą su mediana arba vidurkiu.

Kur buvo reikalinga panaudotas išimčių valdymas (vartotojui įvedant informaciją tikrinama, ar įvestas reikiamas ir taisyklingas skaičius, simbolis, žodis).

Programa apskaičiuoja nuskaitymo, rūšiavimo didėjimo tvarka ir rūšiavimo į dvi grupes trukmes.

Base class zmogus



Derived class studentas



Irodymas, kad realizuota abstrakti klasė zmogus, jos objektų kūrimas negalimas

 